\documentclass{article}
\usepackage[left=1in,right=1in,top=1in,bottom=1in]{geometry}
\usepackage[ruled,vlined,linesnumbered,lined,boxed,commentsnumbered]{algorithm2e}

\setlength\parindent{0pt}

\begin{document}

    \section{Introduction}
    \label{sec:Introduction}
        A table of frequently used symbols: \\ \\
        \begin{tabular}{cl}
            \textbf{Symbol} & \textbf{Representation} \\
            \hline
            g & the input graph \\
            v & the set of nodes in g \\
            e & the set of edges in g \\
            n & a node in v \\
            s & the start node in v \\
            t & the target node in v \\
            f & the number of atoms to conserve (flow to move) \\
            G & the constructed metagraph \\
            V & the set of metanodes in G \\
            E & the set of metaedges in G \\
            N & a metanode in V of the form $\{n_i:x_i\}$ where $n_i \in v$ and
            $x_i$ is the number of atoms at $n_i$\\
            \hline \\
        \end{tabular} \\

        Each $N \in V$ is composed of $i$ many nodes from $v$. A
        metanode takes the form $\{n_i:x_i\}$ where $n_i \in v$ and $x_i$ is
        the number of atoms at $n_i$. The $\sum_0^i x_i = k$ at any given $N$
        ensuring that all atoms are accounted for at each metanode. \\

        The terms \textit{state} and \textit{metanode} are used
        interchangeably. Each metanode is simply a representation of
        where flow is distributed. The state $\{s:f\}$ indicates that
        there is $f$ amount of flow at node $s$. Similarly,
        $\{a:3, b:2\}$ indicates that there are 3 atoms at node
        $a$ and 2 atoms at node $b$.\\

        The goal is to move all $f$ atoms (flow) from node $s$ to node $t$. In
        other words, we seek a path from metanode $\{s:f\}$ to metanode
        $\{t:f\}$ in the metagraph.

        \newpage

    \section{Overview}
    \label{sec:Overview}

        \begin{algorithm}
            \caption{Metagraph Generation}
            \KwIn{g, s, t, f}
            \KwOut{G}

            $S = \{s:f\}$ \tcc{Create the start state}
            $V \gets V \cup S$ \tcc{Add start state to the metagraph}
            $stack.push(S)$\;

            \While{$|stack| \neq 0$}{
                $current \gets stack.pop()$\;
                $partials \gets$ initialize empty list\;

                \For{each inner node, $n$, in $current$}{
                    $innerPartials \gets$ list of all unique ways to move the
                    flow residing at $n$\;
                    $partials.append(innerPartials)$\;
                }

                $nbrs \gets$ all complete states, where each complete state is
                formed by merging together one partial state from each list in
                $partials$\;

                \For{$nbr \in nbrs$}{
                    \If{$nbr$ is valid and $nbr \notin deadset$}{
                        $V \gets V \cup nbr$\;
                        $E \gets E \cup (current, nbr)$\;
                        $nbrCount \gets nbrCount + 1$\;
                        $stack$.\textbf{push}($nbr$)\;
                    }
                }
            }
            \Return $G$\;
        \end{algorithm}

        % \textbf{Input:} g, s, t, f \\
        % \textbf{Output:} G \\
        % \begin{enumerate}
            % \item{Create the start state: $S = \{s:f\}$}
            % \item{Add start state to the metagraph: $V \gets V \cup S$}
            % \item{Push $S$ on to stack: $stack.push(S)$}
            % \item{while the stack is not empty:
                % \begin{enumerate}
                    % \item{$current \gets stack.pop()$}
                    % \item{$partials \gets$ initialize empty list}
                    % \item{for each inner node $n$ in $current$:
                        % \begin{enumerate}
                            % \item{$innerPartials \gets$ list of all unique ways to
                                    % move the flow residing at $n$}
                            % \item{$partials.append(innerPartials)$}
                        % \end{enumerate}
                    % }
                % \item{$nbrs \gets$ all complete states, where each complete
                        % state is formed by merging together one partial state
                        % from each list in $partials$}
                % \item{for each $nbr \in nbrs$}{
                        % \begin{enumerate}
                            % \item{
                                % if $nbr$ is valid and $nbr \notin deadset$}
                                % \begin{enumerate}
                                    % \item{$V \gets V \cup nbr$}
                                    % \item{$E \gets E \cup (current, nbr)$}
                                    % \item{$nbrCount \gets nbrCount + 1$}
                                    % \item{$stack$.\textbf{push}($nbr$)}
                                % \end{enumerate}

                        % \end{enumerate}
                % }
                % \end{enumerate}
            % \item{Return $G$}
            % }
        % \end{enumerate}


        \newpage

    \section{Algorithms}
    \label{sec:Algorithms}

        \begin{algorithm}
            \caption{ConstructMetaGraph}
            \KwIn{$g = (v, e)$: the input graph, $\{s,t\} \in v$: the start and
                target compounds, $k$: flow/number of atoms to conserve}
            \KwOut{$G = (V, E)$: the meta-graph}

            $MG \gets (MV = \emptyset, ME = \emptyset)$ \tcc{initialize new
                metagraph}

            $stack \gets$ initialize empty stack\;

            $start \gets \{s:k\}$ \tcc{Create a metanode with all $k$ atoms at
                the start compound}

            $V \gets V \cup start$ \tcc{Add the start state to the set of
                metanodes}

            $stack.push(start)$ \tcc{Add the start state to the stack to find
                its neighbors}

            $G \gets \textbf{PopulateMetaGraph}(G, stack, target)$ \tcc{Find
                neighboring metanodes to build $G$}

            \Return{G}
        \end{algorithm}

        \begin{algorithm}
            \caption{PopulateMetaGraph}
            \KwIn{$G=(V,E)$: the metagraph, $stack$: stack of metanodes that
                need to be explored, $target$: the metanode state with all $k$
                atoms at node $t$}
            \KwOut{$G=(V,E)$: metagraph with any newly found metanodes added in}

            \While{$|stack| > 0$}{
                $current \gets stack.pop()$ \tcc{Pop off a metanode to explore}
                \If{current = target}{
                    continue \tcc{If we've reached the target, no need to find
                        nbrs}
                }
                \Else{
                    $nbrCount \gets \textbf{IterativeFindMetaNbrs}(current)$\;
                    \If{nbrCount = 0 \textbf{and} current $\neq$ target}{
                        $G \gets \textbf{Prune}(current)$ \tcc{This metanode is
                        a terminus, so remove it from the metagraph}
                    }
                }
            }
            \Return{G}
        \end{algorithm}

        \begin{algorithm}
            \caption{IterativeFindMetaNbrs}
            \KwIn{$current$: the metanode to find nbrs for, $G=(V,E)$: the
                metagraph so far}
            \KwOut{$G=(V,E)$: metagraph with any newly found metanodes added in}

            $partialStates \gets$ initialize new list\;
            \For{$n \in N$}{
                $partialStates$.\textbf{append}(\textbf{RecursiveNbrSearch}(n,
                flow(n), nbrs(n)))\;
            }

            $metanbrs \gets$ \textbf{generateCompleteStates}($partialStates$)\;

            \For{$nbr \in metanbrs$}{
                \If{isValid(nbr) \textbf{and} $nbr \notin deadset$}{
                    $V \gets V \cup nbr$\;
                    $E \gets E \cup (current, nbr)$\;
                    $nbrCount \gets nbrCount + 1$\;
                    $stack$.\textbf{push}($nbr$)\;
                }
            }
            \Return{G}

        \end{algorithm}

        \begin{algorithm}
            \caption{RecursiveNbrSearch}
            \KwIn{$n$: the node we're trying to move flow away from, $flow$:
            the amount of flow residing at $n$ that needs to be moved, $nbrs$:
        the unexplored neighbors of $n$}
            \KwOut{$partialStates$: list of partial states (each state
                represents a unique move of the flow at n to its nbrs)}

            $states \gets$ initialize new list\;

            \If{$|nbrs|$ > 1}{
                $nbr \gets$ nbrs[0]\;
                \For{$i = 0; i \leq min(flow, capacity(parent, nbr)); i++)$}{
                    $partialState \gets$ initialize mapping of nodes (string)
                    to flow (number)\;

                    \If{$i \neq 0$}{
                        partialState.put(nbr, i);
                    }

                    $remaining \gets$ recursiveNbrSearch(n, flow-i, nbrs-nbr)

                    \For{$state \in remaining$}{
                        states.add(mergePartialStates(parialState, state))
                    }
                }
            }
            \If{$|nbrs| = 0$ \textbf{and} flow > 0 \textbf{and} current = target}{
                \tcc{If there is flow remaining, but the current node is the
                target node, the flow can remain stationary}
                state.put(current, flow);
                states.add(state);
            }
            \Else{
                \tcc{If there are no more nbrs and flow remains, no more valid
                    states can be added, so return}
            }

            \Return{states}

        \end{algorithm}


\end{document}
