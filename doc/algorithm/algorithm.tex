\documentclass{article}
\usepackage[left=1in,right=1in,top=1in,bottom=1in]{geometry}
\usepackage[ruled,vlined,linesnumbered,lined,boxed,commentsnumbered]{algorithm2e}

\setlength\parindent{0pt}

\begin{document}

    \section{Introduction}
    \label{sec:Introduction}
        A table of frequently used symbols: \\ \\
        \begin{tabular}{cl}
            \textbf{Symbol} & \textbf{Representation} \\
            \hline
            g & the input graph \\
            v & the set of nodes in g \\
            e & the set of edges in g \\
            n & a node in v \\
            G & the constructed metagraph \\
            V & the set of metanodes in G \\
            E & the set of metaedges in G \\
            N & a metanode in V of the form $\{n_i:x_i\}$ where $n_i \in v$ and
            $x_i$ is the number of atoms at $n_i$\\
            \hline
        \end{tabular} \\

        Each $N \in V$ is composed of $i$ many nodes from $v$. A
        metanode takes the form $\{n_i:x_i\}$ where $n_i \in v$ and $x_i$ is
        the number of atoms at $n_i$. The $\sum_0^i x_i = k$ at any given $N$
        ensuring that all atoms are accounted for at each metanode

    \section{Algorithms}
    \label{sec:Algorithms}

        \begin{algorithm}
            \caption{Construct Meta-Graph}
            \KwIn{$g = (v, e)$: the input graph, $\{s,t\} \in v$: the start and
                target compounds, $k$: flow/number of atoms to conserve}
            \KwOut{$G = (V, E)$: the meta-graph}

            $MG \gets (MV = \emptyset, ME = \emptyset)$ \tcc{initialize new
                metagraph}

            $start \gets \{s:k\}$ \tcc{create a metanode with all $k$ atoms at
                the start compound}

            $V \gets V \cup start$ \tcc{Add the start state to the set of
                metanodes}

        \end{algorithm}

        \begin{algorithm}
            \caption{}
            \KwIn{}
            \KwOut{}

        \end{algorithm}
\end{document}
